\chapter{Concept}\label{ch:concept}
Hoe zou het zijn om in een pretpark de attracties te kunnen doen zonder eindeloos door meanders te hoeven lopen.
We zeggen niet dat je nooit meer hoeft te wachten alleen dat het wachten ander wordt dan je verwacht.
Wat wij voorstellen is het uitbreiden van de fysieke wachtrij in een meer attractie specifieke manier cvan thematisering waardoor de attractie uitgebreider is. De meanders die voorde "pre-show" zitten willen we vervangen door een virtueel wachtrij systeem. Dit systeem moet geen impact hebben op gasten die geen telefoon hebben of geen data beschikbaar hebben.






Een nieuw systeem implementeren in een bestaande infrastructuur is lastig.
Er moet dus gezocht worden naar een implementatie met een licht impact.
Het nieuwe systeem moet zowel voordelen hebben voor de gast als het park.
Waarbij het bij de gast vooral gaat om de beleving van een dag, waarbij het gevoel van VIP voorop staat.
Het park krijgt in ruil voor deze voordelen inzicht in het gedrag van de gasten op het gebied van verplaatsingen door het park.
Een paar belangrijke punten zijn:
\begin{itemize}
    \item Een \textbf{App} moet niet verplicht zijn om in het park een goede dag te hebben.
    Maar het hoeft niet zo te zijn dat de app voordelen mag hebben voor gebruikers.
    \item De \textbf{infrastructurele aanpassingen} moeten gering zijn in die zin dat er minimaal tot weinig aanpassingen gedaan moeten worden in wachtrijen.
    \item Gegevens moeten \textbf{anoniem} worden gegeven kunnen worden. De gegevens die van belang kunnen zijn is de samenstelling van de groep is en eventueel leeftijds groepen van de groepsleden. Andere persoonlijke gegevens zijn niet relevant voor de werking van de app en data vergaring. ID's worden aangemaakt op een niet persoonlijke gegevens en zijn per bezoek anders.
    \item Gegevens die dan \textbf{wel gedeeeld} worden zijn locatie(GPS Of vergelijkbaars) een unieke identifier om een vorm van anonieme tracking mogelijk te maken.
    \item Tracking gegevens worden \textbf{niet realtime} verzonden. Ook dit om ervoor te zorgen dat een id nooit gelinkt kan worden aan groepen en personen. als er een vertraging is van een minuut is het voor het park nog tijd genoeg om actie te ondernemen in operations.
\end{itemize}


Er zijn een aantal infrastructurele requirements die nodig zijn voor het functioneren van het systeem.
\begin{itemize}
    \item wifi positioning: een manier waarop we garanderen dat alleen een positie wordt bepaald binnen het wifi netwerk van het park. en op het momen dat het gezelschap het park verlaat er geen data meer verzameld wordt.
    \item om Wifi positioning te kunnen toepassen da
\end{itemize}
